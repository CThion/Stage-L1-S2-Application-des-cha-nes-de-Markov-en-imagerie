\documentclass[fleqn]{article} %le fleqn c'est pour imposer que sauf indication contraire, tout soit bien alligné sur la gauche (notament \begin{align*}
\usepackage[utf8]{inputenc}
\usepackage{geometry}


\usepackage{amssymb,mathrsfs}
\usepackage{a4}
\usepackage{babel}
\usepackage{amsfonts}
\usepackage{amsmath}
\usepackage{mathptmx}


\usepackage{graphicx}%pour insérer des images
\usepackage{tabto} % pour les tabulations
\usepackage{amsmath,amsfonts,amssymb} %pour les limites
\usepackage{amsmath}

\usepackage{hyperref}
\newcounter{exercice}
\newcounter{question}

\newcommand\exercice{\setcounter{question}{0}\addtocounter{exercice}{1}\vskip 4truemm\noindent\textbf{Exercice \theexercice .} }
\newcommand\quest{\setcounter{sousquestion}{0}\addtocounter{question}{1}\hfill\break\textbf{\thequestion .} }
\newcommand\sousquest{\addtocounter{sousquestion}{1}\hfill\break\hbox{\ \ \ }\textbf{\alph{sousquestion}.} }


\newcommand{\ssi}{\Leftrightarrow} %équivalence
\newcommand{\imp}{\Rightarrow} %implication
\newcommand{\R}{\mathbb{R}} %pour écrir le corps des réels
\newcommand{\N}{\mathbb{N}} %pour écrir le corps des entiers naturels
\newcommand{\ffi}{\varphi}
\newcommand{\Cs}{\mathcal{C}^S}% pour faire un beau C latin
\newcommand{\Cc}{\mathcal{C}^C}
\geometry{hmargin=2.5cm,vmargin=1.5cm}%pour définir les marges

\title{BEP\_DM\_DM1}
\author{Clément Thion, Thomas Besognet}
\date{November 2019}

\begin{document}

\maketitle

\section{Problématiques et objectifs}

\begin{itemize}
	\item Quelle est la probabilité d'avoir une correspondance à 99\% après n itération d'algorithme de recuit simulé? De Gibbs? De Métropolis? \\
		\textit{On va faire un algo de montecarlo dans montecarlo, et introduire une mesure du pourcentage de correspondance}
	\item Peut-on gagner en précision en faisant plusieurs foi un recuit simulé, puis en faisant une "moyenne" des images reçues?\\
		\textit{générer plusieurs image, puis les "additionner"}\\
		\begin{itemize} \item Là encore, comment évolue la proba de correspondance avec le nombre d'image additionné?	\end{itemize}
	\item Peut-on gagner en rapidité dans nos algorithme en parcourant les sites non plus un par un mais n par n ?
\end{itemize}



\end{document}



















