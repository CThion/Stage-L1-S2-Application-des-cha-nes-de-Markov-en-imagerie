\documentclass[fleqn]{article} %le fleqn c'est pour imposer que sauf indication contraire, tout soit bien alligné sur la gauche (notament \begin{align*}
\usepackage[utf8]{inputenc}
\usepackage{geometry}


\usepackage{amssymb,mathrsfs}
\usepackage{a4}
\usepackage{babel}
\usepackage{amsfonts}
\usepackage{amsmath}
\usepackage{mathptmx}


\usepackage{graphicx}%pour insérer des images
\usepackage{tabto} % pour les tabulations
\usepackage{amsmath,amsfonts,amssymb} %pour les limites
\usepackage{amsmath}

\usepackage{hyperref}
\newcounter{exercice}
\newcounter{question}

\newcommand\exercice{\setcounter{question}{0}\addtocounter{exercice}{1}\vskip 4truemm\noindent\textbf{Exercice \theexercice .} }
\newcommand\quest{\setcounter{sousquestion}{0}\addtocounter{question}{1}\hfill\break\textbf{\thequestion .} }
\newcommand\sousquest{\addtocounter{sousquestion}{1}\hfill\break\hbox{\ \ \ }\textbf{\alph{sousquestion}.} }


\newcommand{\ssi}{\Leftrightarrow} %équivalence
\newcommand{\imp}{\Rightarrow} %implication
\newcommand{\R}{\mathbb{R}} %pour écrir le corps des réels
\newcommand{\N}{\mathbb{N}} %pour écrir le corps des entiers naturels
\newcommand{\ffi}{\varphi}
\newcommand{\Cs}{\mathcal{C}^S}% pour faire un beau C latin
\newcommand{\Cc}{\mathcal{C}^C}
\geometry{hmargin=2.5cm,vmargin=1.5cm}%pour définir les marges

\title{BEP\_DM\_DM1}
\author{Clément Thion, Thomas Besognet}
\date{November 2019}

\begin{document}

\maketitle
%=======================================================================================================================================================
\paragraph{démo probabilité générale = somme probabilités locales}
\[
\begin{aligned}
	P(X_s= x_s | X^s= x^s)
		&= \dfrac{P(X_s= x_s, X^s= x^s)}{P(X^s= x^s)}  \text{  formule de Bayes}\\
		&= \dfrac{P(X= x)}{P(X^s= x^s)}\\
		&= \dfrac{e^{-U(x)}}{e^{-U(x^s)} }\\	
		&\ \\ 	
	\text{On a  } U(x^s)
		&= U_{\bar{s}}(x) = \sum_{c\in C, s\notin c}{U_c(x)}\\
		&\ \\
	\text{Et  } U(x) 
		&= \sum_{c\in C}{U_c(x)}\\
		&= \sum_{c\in C, s\in c}{U_c(x)} + \sum_{c\in C, s\notin c}{U_c(x)}\\
		&= U_s(x_s,V_s) + U_{\bar{s}}(x) \\
		&\ \\
	Donc, P(X_s= x_s | X^s= x^s)
		&=\dfrac{exp(-U_s(x_s,V_s) - U_{\bar{s}}(x))}  {exp(-U_{\bar{s}}(x))}\\
		&=\dfrac{exp(-U_s(x_s,V_s)}{exp(-U_{\bar{s}}(x) + U_{\bar{s}}(x))}\\
		&=exp(-U_s(x_s,V_s))   \text{   ce qui justifie l'hypthère markovienne}
\end{aligned}
\]
%=======================================================================================================================================================
\section{Problématiques}

\begin{itemize}
	\item Quelle est la probabilité d'avoir une correspondance à c\% après n itération d'algorithme de recuit simulé? De Gibbs? De Métropolis? \\
		\textit{On va faire un algo de montecarlo dans montecarlo, et introduire une mesure du pourcentage de correspondance}
	\item Quelle est la probabilité d'avoir une correspondance à c\% après k itérations d'algo sans modification d'état?\\
		\textit{cas ou on demande à l'algo de s'arrête uniquement si il aucun changement de site ne se fait, sur k itérations consécutives (attention donc à ce que k ne soit pas trop grand pour que l'on ne tombe pas en boucle infinie, ni trop petit pour que l'algo tourne quand même un minimum}
	\item Peut-on gagner en précision en faisant plusieurs foi un recuit simulé, puis en faisant une "moyenne" des images reçues?\\
		\textit{générer plusieurs image, puis les "additionner"}\\
		\begin{itemize} \item Là encore, comment évolue la proba de correspondance avec le nombre d'images additionnées?	\end{itemize}
	\item Peut-on gagner en rapidité dans nos algorithme en parcourant les sites non plus un par un mais n par n ?\\
		\textit{pour une image de N*N pixels, on peut travailler sur N/2 pixels en même temps sans problème de modification de voisinage sur une même étape}
\end{itemize}

%=======================================================================================================================================================
\section{objectifs pratiques}
\paragraph{Algorithme de Gibbs et Metropolis avec des champs de Markov aléatoires donnés (ising, potts, makovien-gaussien}
\paragraph{Recuit simulé}
\paragraph{Interface graphique pour comparaison des différents algorithmes}

\end{document}



















