% Options for packages loaded elsewhere
\PassOptionsToPackage{unicode}{hyperref}
\PassOptionsToPackage{hyphens}{url}
%
\documentclass[
]{article}
\usepackage{lmodern}
\usepackage{amssymb,amsmath}
\usepackage{ifxetex,ifluatex}
\ifnum 0\ifxetex 1\fi\ifluatex 1\fi=0 % if pdftex
  \usepackage[T1]{fontenc}
  \usepackage[utf8]{inputenc}
  \usepackage{textcomp} % provide euro and other symbols
\else % if luatex or xetex
  \usepackage{unicode-math}
  \defaultfontfeatures{Scale=MatchLowercase}
  \defaultfontfeatures[\rmfamily]{Ligatures=TeX,Scale=1}
\fi
% Use upquote if available, for straight quotes in verbatim environments
\IfFileExists{upquote.sty}{\usepackage{upquote}}{}
\IfFileExists{microtype.sty}{% use microtype if available
  \usepackage[]{microtype}
  \UseMicrotypeSet[protrusion]{basicmath} % disable protrusion for tt fonts
}{}
\makeatletter
\@ifundefined{KOMAClassName}{% if non-KOMA class
  \IfFileExists{parskip.sty}{%
    \usepackage{parskip}
  }{% else
    \setlength{\parindent}{0pt}
    \setlength{\parskip}{6pt plus 2pt minus 1pt}}
}{% if KOMA class
  \KOMAoptions{parskip=half}}
\makeatother
\usepackage{xcolor}
\IfFileExists{xurl.sty}{\usepackage{xurl}}{} % add URL line breaks if available
\IfFileExists{bookmark.sty}{\usepackage{bookmark}}{\usepackage{hyperref}}
\hypersetup{
  pdftitle={Restauration d'image grâce aux chaînes de markov},
  pdfauthor={Clément Thion},
  hidelinks,
  pdfcreator={LaTeX via pandoc}}
\urlstyle{same} % disable monospaced font for URLs
\usepackage[margin=1in]{geometry}
\usepackage{color}
\usepackage{fancyvrb}
\newcommand{\VerbBar}{|}
\newcommand{\VERB}{\Verb[commandchars=\\\{\}]}
\DefineVerbatimEnvironment{Highlighting}{Verbatim}{commandchars=\\\{\}}
% Add ',fontsize=\small' for more characters per line
\usepackage{framed}
\definecolor{shadecolor}{RGB}{248,248,248}
\newenvironment{Shaded}{\begin{snugshade}}{\end{snugshade}}
\newcommand{\AlertTok}[1]{\textcolor[rgb]{0.94,0.16,0.16}{#1}}
\newcommand{\AnnotationTok}[1]{\textcolor[rgb]{0.56,0.35,0.01}{\textbf{\textit{#1}}}}
\newcommand{\AttributeTok}[1]{\textcolor[rgb]{0.77,0.63,0.00}{#1}}
\newcommand{\BaseNTok}[1]{\textcolor[rgb]{0.00,0.00,0.81}{#1}}
\newcommand{\BuiltInTok}[1]{#1}
\newcommand{\CharTok}[1]{\textcolor[rgb]{0.31,0.60,0.02}{#1}}
\newcommand{\CommentTok}[1]{\textcolor[rgb]{0.56,0.35,0.01}{\textit{#1}}}
\newcommand{\CommentVarTok}[1]{\textcolor[rgb]{0.56,0.35,0.01}{\textbf{\textit{#1}}}}
\newcommand{\ConstantTok}[1]{\textcolor[rgb]{0.00,0.00,0.00}{#1}}
\newcommand{\ControlFlowTok}[1]{\textcolor[rgb]{0.13,0.29,0.53}{\textbf{#1}}}
\newcommand{\DataTypeTok}[1]{\textcolor[rgb]{0.13,0.29,0.53}{#1}}
\newcommand{\DecValTok}[1]{\textcolor[rgb]{0.00,0.00,0.81}{#1}}
\newcommand{\DocumentationTok}[1]{\textcolor[rgb]{0.56,0.35,0.01}{\textbf{\textit{#1}}}}
\newcommand{\ErrorTok}[1]{\textcolor[rgb]{0.64,0.00,0.00}{\textbf{#1}}}
\newcommand{\ExtensionTok}[1]{#1}
\newcommand{\FloatTok}[1]{\textcolor[rgb]{0.00,0.00,0.81}{#1}}
\newcommand{\FunctionTok}[1]{\textcolor[rgb]{0.00,0.00,0.00}{#1}}
\newcommand{\ImportTok}[1]{#1}
\newcommand{\InformationTok}[1]{\textcolor[rgb]{0.56,0.35,0.01}{\textbf{\textit{#1}}}}
\newcommand{\KeywordTok}[1]{\textcolor[rgb]{0.13,0.29,0.53}{\textbf{#1}}}
\newcommand{\NormalTok}[1]{#1}
\newcommand{\OperatorTok}[1]{\textcolor[rgb]{0.81,0.36,0.00}{\textbf{#1}}}
\newcommand{\OtherTok}[1]{\textcolor[rgb]{0.56,0.35,0.01}{#1}}
\newcommand{\PreprocessorTok}[1]{\textcolor[rgb]{0.56,0.35,0.01}{\textit{#1}}}
\newcommand{\RegionMarkerTok}[1]{#1}
\newcommand{\SpecialCharTok}[1]{\textcolor[rgb]{0.00,0.00,0.00}{#1}}
\newcommand{\SpecialStringTok}[1]{\textcolor[rgb]{0.31,0.60,0.02}{#1}}
\newcommand{\StringTok}[1]{\textcolor[rgb]{0.31,0.60,0.02}{#1}}
\newcommand{\VariableTok}[1]{\textcolor[rgb]{0.00,0.00,0.00}{#1}}
\newcommand{\VerbatimStringTok}[1]{\textcolor[rgb]{0.31,0.60,0.02}{#1}}
\newcommand{\WarningTok}[1]{\textcolor[rgb]{0.56,0.35,0.01}{\textbf{\textit{#1}}}}
\usepackage{graphicx}
\makeatletter
\def\maxwidth{\ifdim\Gin@nat@width>\linewidth\linewidth\else\Gin@nat@width\fi}
\def\maxheight{\ifdim\Gin@nat@height>\textheight\textheight\else\Gin@nat@height\fi}
\makeatother
% Scale images if necessary, so that they will not overflow the page
% margins by default, and it is still possible to overwrite the defaults
% using explicit options in \includegraphics[width, height, ...]{}
\setkeys{Gin}{width=\maxwidth,height=\maxheight,keepaspectratio}
% Set default figure placement to htbp
\makeatletter
\def\fps@figure{htbp}
\makeatother
\setlength{\emergencystretch}{3em} % prevent overfull lines
\providecommand{\tightlist}{%
  \setlength{\itemsep}{0pt}\setlength{\parskip}{0pt}}
\setcounter{secnumdepth}{-\maxdimen} % remove section numbering

\title{Restauration d'image grâce aux chaînes de markov}
\author{Clément Thion}
\date{06/2020}

\begin{document}
\maketitle

\hypertarget{moduxe9lisation-dune-image}{%
\section{Modélisation d'une image}\label{moduxe9lisation-dune-image}}

Pour la suite on utilise les notations suivantes:\\
- s : un pixel de l'image, avec \(x_s\) sa valeur (-1 ou 1 dans notre
cas)\\
- \(V_s\) : le voisinage de s\\
- \(U(x_s)\) : le potentiel de s pour la valeur \(x_s\). On parle aussi
d'énergie.\\
- \(X_s\) : la variable aléatoire décrivant l'état de s. Pour nous
\(X_s\in\{-1,1\}\)\\
- X : la variable aléatoire décrivant la configuration de l'image\\
-
\fbox{$\tilde X^s=$la variable aléatoire décrivant la configuration de l'image privée du pixel $s$.}

\hypertarget{hypothuxe8se-markovienne}{%
\subsection{Hypothèse markovienne}\label{hypothuxe8se-markovienne}}

A priori pour ``deviner'' quelle est la bonne valeur d'un pixel dans une
image données, le mieux à faire est de considérer tous le reste de
l'image, donc la valeur de chacun des autres pixels de l'image, et alors
d'en tirer une conclusion sur \(P(X_s=x_s)\). On fait cependant
l'hypothèse que connaître le voisinage proche de s est équivalent à
connaître toute l'image, soit \(P(X_s=x_s|X)=P(X_s=x_s|V_s)\).
\fbox{\vbox{Il faut écrire $P(X_s=x_s|\tilde X^s)=P(X_s=x_s|V_s)$. En effet si on écrit $P(X_s=x_s|X)$, cette probabilité vaut $0$ ou $1$, puisque $X$ contient la valeur du pixel $s$! (elle vaut $1$ si la valeur du pixel 
$s$ dans la configuration d'image $X$ vaut $x_s$ et $0$ sinon).}}

\hypertarget{le-moduxe8le-dising}{%
\subsection{Le modèle d'Ising}\label{le-moduxe8le-dising}}

Le modèle d'Ising nous fournit une formule pour calculer le potentiel
d'un pixel, pour un état donné, en fonction de son voisinage. On a:

\[U(x_s) = -\beta x_s \sum_{t\in V_s}{x_t} -B\sum_{s\in S}{x_s}\]

On verra une de ses limites dans la partie sur l'estimation de
\(\beta\).

\hypertarget{adapation-uxe0-la-restauration-dimage}{%
\subsection{Adapation à la restauration
d'image}\label{adapation-uxe0-la-restauration-dimage}}

Pour la restauration d'image, on comprend qu'il manque quelque chose au
modéle d'Ising. Tel qu'il est, avec un algorithme de metropolis, notre
image tendrait irrémédiablement vers l'image blanche, qui est bien
l'image de plus faible énergie dans l'absolu. Il nous faut passer à la
probabilité conditionnée par la configuration de départ que l'on note
\(X^0\): \(P(X_s=x_s|X^0,V_s)\). Pour se faire on ajoute un paramètre
pour ternir compte de l'image bruitée de départ, de sorte que la
configuration de plus basse énergie ne soit plus l'image blanche mais
bien l'image que l'on souhaite restaurer. Ainsi on modifie \(U(x_s)\)
par:
\[ U(x_s) = -\beta x_s \sum_{t\in V_s}{x_t} -B\sum_{s\in S}{x_s} -\alpha x_s^0\]
avec \(x_s^0\) la valeur du site s de l'image bruitée de départ.

\hypertarget{des-algorithmes-pour-la-restauration-dimage}{%
\section{Des algorithmes pour la restauration
d'image}\label{des-algorithmes-pour-la-restauration-dimage}}

\hypertarget{lalgorithme-de-metropolis}{%
\subsection{L'algorithme de
Metropolis}\label{lalgorithme-de-metropolis}}

\hypertarget{metropolisising}{%
\subsubsection{metropolisIsing}\label{metropolisising}}

\textbf{Le paramètre alpha} détermine l'importance donnée à l'image
bruitée initiale dans la probabilité conditionnelle.\\
- Si alpha est trop grand devant \(\beta\), les valeurs les plus
probables seront toujours celles de l'image bruitée, et donc
l'algorithme ne fera rien.\\
- S'il est trop petit voir nul, la configuration initiale va être
oubliée itération après itération, et on va tendre vers l'image branche,
soit donc la configuration d'énergie minimale parmi toutes les
configurations possibles.

\textbf{Le paramètre \(\beta\)}

\textbf{Le calcul du voisinage} se fait grâce à la fonction Vstar,
détaillée plus bas. Pour avoir le voisinage standard (les quatres pixels
haut bas gauche droite), il suffit de mettre 1 comme valeur pour L.

\textbf{Le parcours des sites} peut se faire soit aléatoirement, soit
ligne par ligne.

\hypertarget{luxe9chantilloneur-de-gibbs}{%
\subsection{L'échantilloneur de
Gibbs}\label{luxe9chantilloneur-de-gibbs}}

\hypertarget{gibbsising}{%
\subsubsection{gibbsIsing}\label{gibbsising}}

\hypertarget{le-recuit-simuluxe9}{%
\subsection{Le recuit simulé}\label{le-recuit-simuluxe9}}

\hypertarget{estimation-du-paramuxe8tre-beta}{%
\section{\texorpdfstring{Estimation du paramètre
\(\beta\)}{Estimation du paramètre \textbackslash beta}}\label{estimation-du-paramuxe8tre-beta}}

L'algorithme de recuit précédent marche plutôt bien pour des images
pleines (rectangle, losange), mais si on éssaie de restaurer une image
en plusieurs parties, une succession de bandes par exemple, ou un
damier, l'algorithme ne fonctionne plus du tout.

\begin{Shaded}
\begin{Highlighting}[]
\CommentTok{\#metropolisIsing("ligne", 64, 1, 0, 1, 10, 10\^{}4, 0.15)}
\end{Highlighting}
\end{Shaded}

\hypertarget{distinction-des-uxe9nergies-verticales-et-horizontales}{%
\subsection{Distinction des énergies verticales et
horizontales}\label{distinction-des-uxe9nergies-verticales-et-horizontales}}

Une idée que l'on peut avoir pour palier au problème est de séparer les
calcules de potentiels entre la verticale et l'horizontale, en
appliquant à chaque terme un facteur beta dédié à une orientation,
\(\beta_{horizontal}\) et \(\beta_{vertical}\). On a ainsi:
\(U(x_s)= -x_s[\beta_h(x_d+x_g)+\beta_v(x_h+x_b)+\alpha x_s^0]\) avec
\(x_d,x_g,x_h,x_b\) les valeurs des pixels voisins du site s
respectivement à droite, gauche, haut, bas. On modifie l'algorithme de
recuit pour essayer:

On peut ainsi avoir des signes différents pour \(\beta_h\) et
\(\beta_v\), et donc privilégier les lignes pour une direction et les
alternances blanc/noir dans l'autre direction. Pour l'exemple précédent
avec des lignes horizontales, on va choisir \(\beta_h > 0\) pour
privilégier les lignes sur l'horizontale et \(\beta_v < 0\) pour
privilégier l'alternance des pixels sur la verticale.

\begin{Shaded}
\begin{Highlighting}[]
\CommentTok{\#algo qui va bien}
\end{Highlighting}
\end{Shaded}

\hypertarget{voisinage-de-taille-variable}{%
\subsection{Voisinage de taille
variable}\label{voisinage-de-taille-variable}}

\hypertarget{amuxe9lioration-de-la-rapidituxe9-des-algorithmes}{%
\section{Amélioration de la rapidité des
algorithmes}\label{amuxe9lioration-de-la-rapidituxe9-des-algorithmes}}

Depuis le début, on fixe le nombre d'itérations de l'algorithme (souvent
à \(10^4\) ou \(10^5\)). Cependant, on peut supposer que parfois il ne
soit pas nécessaire de faire autant d'itération pour arriver à un
résultat correct. De plus, le bruitage est réalisé aléatoirement, et
donc d'une exécution de d'algorithme à une autre, le nombre d'itération
nécessaire pour restaurer l'image ne sera pas forcément le même.

Une manière de rendre le nombre d'itération variable serait de le
conditionner au nombre de parcours de site successifs sans changement de
valeurs. On pourra ainsi déclarer arbitrairement que, dès que
l'algorithme réalise dix itérations successive sans changement de
valeur, il s'arrête. C'est ce que l'on fait dans l'algorithme de
metropolis suivant:

\begin{Shaded}
\begin{Highlighting}[]
\CommentTok{\#algo qui va bien}
\end{Highlighting}
\end{Shaded}

\fbox{\vbox{\textbf{Une idée :} pour aller encore plus vite, au lieu d'attendre dix itérations successives sans aucune modification, on pourrait arrêter l'algorithme dès que, au cours des dix dernières itérations, il y a au plus 1 modification. On peut jouer sur ces deux paramètres (taille de la "fenêtre coulissante", ici 10, et nombre maximum de changements, 0 dans votre exemple, et ici 1).}}

\hypertarget{evolution-de-la-restauration-en-fonction-dun-nombre-dituxe9rations}{%
\section{Evolution de la restauration en fonction d'un nombre
d'itérations}\label{evolution-de-la-restauration-en-fonction-dun-nombre-dituxe9rations}}

Avec des algorithmes dont le nombre d'itération varie, on peut se
demander quelle est la probabilité d'avoir une image restaurée à un H\%,
après n itérations consécutives sans changements.

\hypertarget{probabilituxe9-davoir-la-configuration-duxe9nergie-minimal}{%
\subsection{Probabilité d'avoir la configuration d'énergie
minimal}\label{probabilituxe9-davoir-la-configuration-duxe9nergie-minimal}}

\hypertarget{probabilituxe9-de-succuxe8s-pour-un-pixel}{%
\subsubsection{Probabilité de succès pour un
pixel}\label{probabilituxe9-de-succuxe8s-pour-un-pixel}}

\hypertarget{probabilituxe9-de-succuxe8s-pour-une-configuration-compluxe8te}{%
\subsubsection{Probabilité de succès pour une configuration
complète}\label{probabilituxe9-de-succuxe8s-pour-une-configuration-compluxe8te}}

\hypertarget{probabilituxe9-davoir-restauruxe9-limage-uxe0-h}\label{probabilituxe9-davoir-restauruxe9-limage-uxe0-h}}

\end{document}
